% !TEX root =../LibroTipoETSI.tex
%El anterior comando permite compilar este documento llamando al documento raíz
\chapter{Introducción}\label{chp-01}

\section{Motivación}

\lettrine[lraise=-0.1, lines=2, loversize=0.2]{H}oy en día es innegable que
las tecnologías enfocadas al IoT (Internet de las
cosas) están en pleno auge. La tendecia es conectar todo lo que se pueda a
Internet para así hacerlo inteligente.

Imaginemos que un fabricante decide crear un dispositivo, digamos que desarrolla
un sensor de temperatura para que un usuario pueda monitorizar la temperatura
de una habitación mediante una interfaz web o app móvil. Al poco de plantear
alguna solución encontrará que desarrollar toda una arquitectura para poder
comunicar su sensor de temperatura con la aplicación web puede ser un proceso
complejo. Será necesario obtener datos de miles de sensores, tratar los datos,
almacenarlos, poder obtenerlos desde la aplicación web o móvil, etc.

Mientras que una organización de gran tamaño con suficientes recursos puede
abordar el problema, para pequeñas compañías o startups puede suponer un
handicap. El desarrollo de un backend sobre el que se apoye su producto puede
ser una tarea que termine por hacer que el proyecto sea inviable, ya que no
permite al desarrollador centrar todos sus esfuerzos en desarrollar su producto
y le obliga a gastar recursos en construir y mantener su backend.

En el panorama actual existe una gran alternativas a la hora de elegir las
diferentes tecnologías que compondrán el sistema. El mero hecho de realizar
un estado del arte ya supone un esfuerzo. Para solucionar cada pequeño problema
podemos encontrar una gran variedad de soluciones y a la hora de la integración
de las diferentes partes pueden surgir más problemas.

Todo esto no hace más que suponer una barrera para los fabricantes que puede
desembocar en que el proyecto nunca sea llevado a cabo.
